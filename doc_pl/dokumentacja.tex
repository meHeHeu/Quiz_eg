\documentclass[12pt, a4paper]{article}

\usepackage{polski}
\usepackage[utf8]{inputenc}
\usepackage{amsmath}
\usepackage{amssymb}
\usepackage{hyperref}
\usepackage{graphicx}

\usepackage{listings}
\lstset{
%numbers=left
}

\usepackage{fancyhdr}
%\pagestyle{fancy}
%\lhead{Michał Kloc}
%\rhead{Quiz}

\usepackage{geometry}
%\newgeometry{tmargin=4.5cm, bmargin=2.5cm, lmargin=2.5cm, rmargin=2.5cm}
\newgeometry{tmargin=2.5cm, bmargin=2.5cm, lmargin=2.5cm, rmargin=2.5cm}

\title{Quiz - projekt z przedmiotu Mobilne Interfejsy Multimedialne}
\author{Michał Kloc}
\date{}

\PassOptionsToPackage{hyphens}{url}\usepackage{hyperref}


\begin{document}

\begin{titlepage}
	\centering
	\includegraphics[scale=0.2]{imgs/logo_polsl.jpg}\par\vspace{1cm}
	{\scshape\LARGE Politechnika Śląska \par}
	\vspace{1cm}
	{\scshape\Large Mobilne Interfejsy Multimedialne.\par}
	\vspace{1.5cm}
	{\huge\bfseries Quiz\par}
	\vspace{2cm}
	{\Large\itshape Michał Kloc\par}
	\vfill
	{\large \today\par}
\end{titlepage}

\tableofcontents
\setcounter{page}{0}
\newpage


\part{Użytkowanie}

\section{Opis aplikacji}
Aplikacja pozwala na rozwiązywanie wcześniej przygotowanych quizów.

\section{Wymagania systemowe}
Program był testowany na telefonie \textbf{Galaxy A5} z systemem \textbf{Android 7.0}.\\
Do skompilowania aplikacji wykorzystano środowisko \textbf{Android-Studio 3.0.1} w systemie GNU/Linux, dystrybucji Mint.

\section{Instrukcja obsługi}

\subsection{Baza pytań}
Po uruchomieniu, aplikacja przeszukuje poniższe lokalizacje w celu załadowania listy pytań:
\begin{itemize}
\item Quiz
\item Android/data/Quiz
\item Documents/data/Quiz
\end{itemize}
W przypadku nie znalezienia żadnej bazy pytań, program kończy swoje działanie.\\
W menu jest możliwość wyboru dostępnych quizów.
\newpage

\subsection{Format bazy pytań}
Pliki zawierające pytania do quizu muszą być następującego formatu:
\begin{lstlisting}
<test>

	<question>
		<body>    TRESC PYTANIA      </body>
		<answer>  POPRAWNA ODPOWIEDZ </anwer>
		<banswer> ZLA ODPOWIEDZ      </banswer>
		<banswer> ZLA ODPOWIEDZ      </banswer>
		<banswer> ZLA ODPOWIEDZ      </banswer>
		...
	</question>

	<question>
	...
	</question>

	...

</test>
\end{lstlisting}
Baza pytań znajduje się w elemencie nadrzędnym \textbf{test}.\\
Każde pytanie reprezentowane jest przez element \textbf{question}.\\
\textbf{Question} musi posiada elementy:
\begin{itemize}
\item \textbf{body} - treść pytania
\item \textbf{answer} - poprawną odpowiedź
\item \textbf{banswer} - niepoprawne odpowiedzi (powinno być więcej niż jedna)
\end{itemize}
Oprócz tego element \textbf{test} może zawierać dodatkowe tagi opisujące pytania (np. autor, źródło) - w tej wersji aplikacji zostaną one zignorowane.
\textbf{Każdy inny sposób opisania pytań powoduje niezdefiniowane zachowanie.}

\subsection{Quiz}
Quiz rozwiązuje się klikając przyciski z odpowiedziami. Pytanie oraz wynik znajdują się powyżej przycisków wyboru.\\
Każda poprawna odpowiedź gwarantuje 1pt.\\
Quiz kończy się po odpowiedzeniu na ostatnie pytanie.\\

\section{Bugi}
Dotychczasowe testy nie wykazały żadnych błędów.
\newpage


\part{Opinie użytkowników}
\section{Grupa docelowa}
Za grupę docelową wybrałem użytkowników w wieku 10-60 lat.
\section{Opinie}
\begin{itemize}
\item Pani 50-55\\
Raczej się podobało. Przejrzysty interfejs. Kolorystyka ``trochę jak na pogrzeb'' - mogłaby być bardziej wesoła.
\item Pan 20-25\\
Jedno zastrzeżenie: przyciski powinny być trochę lepiej odróżnialne od siebie (większy dystans, cień, ramka lub coś w tym stylu).
\item Chłopak 15-20\\
Proponowane zmiany/poprawki:
\begin{itemize}
\item Dodać przycisk powrotu do menu.
\item Przy naciśnięciu przycisku \textbf{back} poprosić o potwierdzenie.
\item Tekst w przyciskach wyboru powinien być większy i bardziej na środku.
\item Postarać się lepiej zagospodarować przestrzeń między pytaniem a przyciskami.
\item Napisy ``Dobrze/Źle'' się zlewają z przyciskami wyboru odpowiedzi.
\item Brzydki, nieczytelny dialog końcowy. Propozycja stworzenia ekranu końcowego.
\item Nie dawać bezpośrednio listy quizów w głównym menu. Stworzyć podmenu: nowa gra, opcje, autorzy, itp.
\item Kolorystyka ustalana w quizie.
\end{itemize}
\item Autor - w miesiąc po zakończeniu projektu.\\
Od razu rzuciło mi się w oczy złe miejsce pojawiania się etykiety Dobrze/Źle. Zanim zakończyłem projekt, kolorystyka była znacznie jaśniejsza, dlatego też nie dawało się to tak we znaki.\\
Co do kolorystki: można by ustalać wygląd aplikacji w zależności od quizu (jego tematyki), jaki jest rozwiązywany.\\
Nie jestem pewien co do przyciku ``wróć'' i głównego menu z opcjami - cierpi na tym prostota aplikacji. Przycisk opcji można by umieścić na pasku na górze.

Poza tym zgadzam się z proponowanymi zmianami wszystkich opiniujących.
\end{itemize}
\newpage


\part{Opis projektu}
\section{Lista plików}
\begin{itemize}
\item app/src/main/\textbf{AndroidManifest.xml}
\item app/src/main/res/layout/\textbf{activity\_main.xml}
\item app/src/main/res/layout/\textbf{activity\_quiz.xml}
\item app/src/main/res/layout/\textbf{question\_item.xml}
\item app/src/main/res/layout/\textbf{quiz\_item.xml}
\item app/src/main/res/layout/\textbf{toolbar.xml}
\item app/src/main/res/menu/\textbf{menu\_main.xml}
\item app/src/main/res/values/\textbf{colors.xml}
\item app/src/main/res/values/\textbf{dimens.xml}
\item app/src/main/res/values/\textbf{strings.xml}
\item app/src/main/res/values/\textbf{styles.xml}
\item app/src/main/res/values-pl/\textbf{styles.xml}
\item app/src/main/java/com/example/quiz\_eg/\textbf{utils/Constants.java}
\item app/src/main/java/com/example/quiz\_eg/\textbf{utils/FileUtils.java}
\item app/src/main/java/com/example/quiz\_eg/\textbf{utils/RandomUtils.java}
\item app/src/main/java/com/example/quiz\_eg/\textbf{Questions.java}
\item app/src/main/java/com/example/quiz\_eg/\textbf{MainActivity.java}
\item app/src/main/java/com/example/quiz\_eg/\textbf{QuizActivity.java}
\end{itemize}
\newpage

\section{Opis plików}

\subsection{AndroidManifest.xml}
Ustawienia aplikacji.\\
\begin{itemize}
\item Orientacja pionowa aktywności.
\item Wymagane pozwolenie na odczytywanie pamięci.
\item \textbf{QuizActivity} ma aktywność rodzicielską \textbf{MainActivity}.
\end{itemize}

\subsection{układy}
\begin{itemize}
\item \textbf{activity\_main.xml} - układ menu głównego
\item \textbf{activity\_quiz.xml} - układ quizu
\item \textbf{question\_item.xml} - układ (wygląd) elementu w liście pytań
\item \textbf{quiz\_item.xml} - układ (wygląd) elementu w liście quizów
\item \textbf{toolbar.xml} - wygląd paska narzędzi
\end{itemize}

\subsection{tekst}
Plik \textbf{value/strings.xml} zawiera tekst w interfejsie w języku angielskim.\\
Plik \textbf{value-pl/strings.xml} zawiera tłumaczenie powyższego pliku na język polski.

\subsection{Wygląd interfejsu}
\begin{itemize}
\item \textbf{colors.xml} - kolor elementów układów
\item \textbf{dimens.xml} - rozmiar elementów układów
\item \textbf{styles.xml} - styl elementów
\end{itemize}

\subsection{Paczka utils}
\textbf{Nie} można \textbf{instancjonować} klas w tej paczce. \textbf{Nie} można też ich \textbf{rozszerzać}. Zawiera ona funkcje i zmienne pomocnicze.
\subsubsection{Constants}
Globalne stałe dla projektu.
\begin{itemize}
\item\begin{lstlisting}
public static final int STORAGE_PERMISSION_CODE;
\end{lstlisting}
Służy do obsługi uprawnień aplikacji.
\item\begin{lstlisting}
public static final String EXTERNAL_STORAGE_DIRECTORY;
\end{lstlisting}
Ścieżka do katalogu domowego.
\item\begin{lstlisting}
public static final String QUIZ_EXTRA;
\end{lstlisting}
Klucz do przenoszenia plików z quizem między aktywnościami.
\item\begin{lstlisting}
public static final String[] QUIZ_PATHS;
\end{lstlisting}
Ściężki (względem katalogu domowego) w których aplikacja będzie wyszukiwać quizów.
\item\begin{lstlisting}
public static final String QUIZ_FILE_EXTENSIONS;
\end{lstlisting}
Rozszerzenie plików, w których poszukiwane będą quizy.
\item\begin{lstlisting}
public static final String PERMISSIONS_REQUIRED;
\end{lstlisting}
List pozwoleń, jakich aplikacja wymaga.
\end{itemize}

\subsection{FileUtils}
Operacje na plikach.
\begin{lstlisting}
public static List<File> get(File root, String[] extensions);
\end{lstlisting}
Jeżeli \textbf{root} jest folderem, zwraca listę wszystkich plików w tym folderze i jego podkatalogach, które mają roszerzenie w tablicy \textbf{extensions}.

\subsection{RandomUtils}
Zawiera narzędzia do losowania.
\begin{lstlisting}
public static <T> void shuffleList(List<T> list);
public static <T> void shuffleArray(T[] array);
\end{lstlisting}
Wymieszaj listę / tablicę.

\section{Questions.java}
Służy do wczytywania, przechowywania i udostępniania danych aktywności.\\
Format wczytywanych danych opisany jest w części Użytkowanie, sekcja Instrukcja obsługi, Format bazy pytań.
\begin{itemize}
\item\begin{lstlisting}
Questions(File file);
\end{lstlisting}
Konstruktor. Wczytuje bazę pytań z pliku \textbf{file}.
\item\begin{lstlisting}
Questions(String... fileNames);
\end{lstlisting}
Konstruktor. Wczytuje bazę pytań z plików o nazwach \textbf{fileNames}.
\item\begin{lstlisting}
static class Question;
\end{lstlisting}
Zapewnia interfejs do odzyskiwania informacji o pytaniach.
\begin{itemize}
\item\begin{lstlisting}
public final String body;
\end{lstlisting}
Treść pytania.
\item\begin{lstlisting}
public final String answer;
\end{lstlisting}
Odpowiedź prawidłowa.
\item\begin{lstlisting}
public final List<String> banswer;
\end{lstlisting}
Lista nieprawidłowych odpowiedzi.
\item\begin{lstlisting}
public Question(
	String body,
	String answer,
	List<String> badanswer
);
\end{lstlisting}
Konstruktor.
\end{itemize}

\item\begin{lstlisting}
Question getQuestion(int i);
\end{lstlisting}
Zwraca pytanie numer \textbf{i}.
\item\begin{lstlisting}
int number();
\end{lstlisting}
Zwraca liczbę załadowanych pytań.
\end{itemize}

\section{MainActivity.java}
Opisuje główne menu.\\
Sprawdza, czy aplikacja posiada odpowienie uprawnienia.\\
Wczytuje listę baz z quizami.\\

\section{QuizActivity.java}
Opisuje aktywność przeprowadzającą quiz.\\

\end{document}

