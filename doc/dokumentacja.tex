\documentclass[12pt, a4paper]{article}
\usepackage{polski}
\usepackage[utf8]{inputenc}
\usepackage{amsmath}
\usepackage{amssymb}
\usepackage{hyperref}
\usepackage{graphicx}
\usepackage{listings}
\lstset{
%numbers=left
}
\usepackage{fancyhdr}
%\pagestyle{fancy}
%\lhead{Alan Andrzejak, Aleksy Dziarmaga, Michał Kloc}
%\rhead{Dokumentacja projektu Mobilna edukacja}
\usepackage{geometry}
%\newgeometry{tmargin=4.5cm, bmargin=2.5cm, lmargin=2.5cm, rmargin=2.5cm}
\newgeometry{tmargin=2.5cm, bmargin=2.5cm, lmargin=2.5cm, rmargin=2.5cm}
\title{08. Mobilna edukacja: wykład oraz kolokwium w postaci testu jednokrotnego wyboru - projekt}
\author{Alan Andrzejak, Aleksy Dziarmaga, Michał Kloc}
\date{}
\usepackage{tikz}
\usetikzlibrary{shapes.geometric, arrows}
\tikzstyle{startstop} = [rectangle, rounded corners, minimum width=3cm, minimum height=1cm,text centered, draw=black, fill=red!30]
\tikzstyle{io} = [trapezium, trapezium left angle=70, trapezium right angle=110, minimum width=3cm, minimum height=1cm, text centered, draw=black, fill=blue!30]
\tikzstyle{process} = [rectangle, minimum width=3cm, minimum height=1cm, text centered, draw=black, fill=orange!30]
\tikzstyle{decision} = [diamond, minimum width=3cm, minimum height=1cm, text centered, draw=black, fill=green!30]
\tikzstyle{arrow} = [thick,->,>=stealth]
\PassOptionsToPackage{hyphens}{url}\usepackage{hyperref}
\begin{document}
\begin{titlepage}
	\centering
	\includegraphics[scale=0.2]{imgs/logo_polsl.jpg}\par\vspace{1cm}
	{\scshape\LARGE Politechnika Śląska \par}
	\vspace{1cm}
	{\scshape\Large 08. Mobilna edukacja:\par}
	\vspace{1.5cm}
	{\huge\bfseries wykład oraz kolokwium w postaci testu jednokrotnego wyboru.\par}
	\vspace{2cm}
	{\Large\itshape Alan Andrzejak, Aleksy Dziarmaga, Michał Kloc\par}
	\vfill
	{\large \today\par}
\end{titlepage}
\tableofcontents
\setcounter{page}{0}
\newpage
\part{Użytkowanie}
\section{Opis aplikacji}
Na obecnym etapie projektu "Mobilna edukacja", dostępna jest aplikacja "Quiz", pozwalająca na przetestowanie własnej wiedzy mobilnie na przygotowany wcześniej temat.
\section{Wymagania systemowe}
Program był testowany na telefonie \textbf{Galaxy A5} z systemem \textbf{Android 7.0}.\\
Do skompilowania aplikacji wykorzystano środowisko \textbf{Android-Studio 3.0.1} w systemie GNU/Linux, dystrybucji Mint.
\section{Instrukcja obsługi}
\subsection{Baza pytań}
Po uruchomieniu, aplikacja przeszukuje poniższe lokalizacje w celu załadowania listy pytań:
\begin{itemize}
\item Quiz\_eg
\item Android/data/Quiz\_eg
\item Documents/data/Quiz\_eg
\end{itemize}
W przypadku nie znalezienia żadnego pliku .xml program zakańcza swoje działanie.
\newpage
\subsection{Format bazy pytań}
Pliki zawierające pytania do quizu muszą być następującego formatu:
\begin{lstlisting}
<test>

	<question>
		<body>    TRESC PYTANIA      </body>
		<answer>  POPRAWNA ODPOWIEDZ </anwer>
		<banswer> ZLA ODPOWIEDZ      </banswer>
		<banswer> ZLA ODPOWIEDZ      </banswer>
		<banswer> ZLA ODPOWIEDZ      </banswer>
	</question>

	<question>
	...
	</question>

	...

</test>
\end{lstlisting}
Plik .xml musi składać się z korzenia o nazwie \textbf{test} i zawierać elementy o nazwie \textbf{question}.\\
Każdy element \textbf{question} musi zawierać po jednym elemencie \textbf{body} i \textbf{answer}, oraz 3 elementy \textbf{banswer}.
Oprócz tego element \textbf{test} może zawierać dodatkowe tagi opisujące pytania - zostaną one zignorowane.
\textbf{Każdy inny sposób opisania pytań może spowodować niezdefiniowane zachowanie.}
\subsection{Quiz}
Quiz rozwiązuje się klikając odpowiednie przyciski. Pytanie oraz ilość punktów znajdują się powyżej przycisków wyboru.\\
Każda poprawna odpowiedź gwarantuje 1pt.\\
Quiz kończy się po odpowiedzeniu na ostatnie pytanie.\\
Aplikacja zapewnia możliwość ponownego przejścia quizu.
\section{Bugi}
Dotychczasowe testy nie wykazały żadnych niezgodności.
\newpage
\part{Opis projektu}
\section{Opis ważniejszych plików źródłowych}
\begin{itemize}
\item app/src/main/\textbf{AndroidManifest.xml}
\item app/src/main/res/layout/\textbf{activity\_main.xml}
\item app/src/main/res/values/\textbf{strings.xml}
\item app/src/main/java/com/example/quiz\_eg/\textbf{MyUtils.java}
\item app/src/main/java/com/example/quiz\_eg/\textbf{Questions.java}
\item app/src/main/java/com/example/quiz\_eg/\textbf{MainActivity.java}
\end{itemize}
\subsection{AndroidManifest.xml}
Ustawienia aplikacji.\\
Ustawiono orientację pionową na stałe oraz uprawnienia, jakich aplikacja zarząda.
\subsection{activity\_main.xml}
Opis wyglądu interfejsu aplikacji.
\subsection{strings.xml}
Wszelkie stałe etykiety i wiadomości.
\subsection{MyUtils.java}
Klasa zapewniająca dodatkowe narzędzia.\\
Spis metod:\\
\begin{itemize}
\item
\begin{lstlisting}
public static <T> void ShuffleArray(T[] array);
\end{lstlisting}
Przestawia losowo (tasuje) elementy tablicy.\\
Potrzebne do losowania kolejności pytań oraz kolejności wyświetlania odpowiedzi.
\item\begin{lstlisting}
public static List<File>
FileListByExtension(
File root, String[] extensions
);
\end{lstlisting}
Zwraca listę wszystkich plików w podścieżce \textbf{root} o rozszerzeniach w tablicy \textbf{extensions}.\\
Potrzebne do wyszukania plików .xml.
\end{itemize}
\subsection{Questions.java}
Zawiera klasę \textbf{Questions} służącą jako kontener na pytania.\\\\
Członkowie klasy:
\begin{itemize}
\item\begin{lstlisting}
class Question;
\end{lstlisting}
Gwarantuje interfejs dla części składowych pytania.
\item\begin{lstlisting}
Questions(String ...fileNames);
\end{lstlisting}
Konstruktor, przeszukuje podane lokalizacje po pliki .xml, parsuje je i wypełnia kontener \textbf{question}.
\item\begin{lstlisting}
Question getQuestion(int i);
\end{lstlisting}
Zwraca pytanie (element typu \textbf{Question}) o indeksie \textbf{i}.
\item\begin{lstlisting}
int amount();
\end{lstlisting}
Zwraca ilość pytań w kontenerze \textbf{question}.
\end{itemize}
Prywatni członkowie klasy:
\begin{itemize}
\item\begin{lstlisting}
private Question[] question;
\end{lstlisting}
Kontener trzymający wszystkie wczytane przez konstruktor pytania.
\item\begin{lstlisting}
private class TestXmlParser;
\end{lstlisting}
Klasa odpowiedzialna za przetworzenie pliku .xml na dane ładowane do obiektów typu \textbf{Qestion}.
\end{itemize}
\subsubsection{Klasa Question}
Interfejs dający dostęp do elementów pytania.\\\\
Dostępne pola:
\begin{itemize}
\item\begin{lstlisting}
public final String body;
\end{lstlisting}
Treść pytania.
\item\begin{lstlisting}
public final String answer;
\end{lstlisting}
Odpowiedź na pytanie.
\item\begin{lstlisting}
public final String[] banswer;
\end{lstlisting}
Złe odpowiedzi.
\end{itemize}
Konstruktor:
\begin{lstlisting}
public
Question(
String body,
String answer,
String ...banswer
);
\end{lstlisting}
\subsubsection{Klasa TestXmlParser}
\textbf{TestXmlParser} jest klasą prywatną na użytek klasy \textbf{Questions}.\\
Służy do parsowania strumienia z pliku .xml do obiektu klasy \textbf{Question}.\\
Parsowanie wykonuje się metodą:
\begin{lstlisting}
List<Question> parse(FileInputStream fileInputStreams)
                throws XmlPullParserException, IOException;
\end{lstlisting}
Metody prywatne:
\begin{lstlisting}
private List<Question> readTest(XmlPullParser parser)
        throws XmlPullParserException, IOException;

private Question readQuestion(XmlPullParser parser)
        throws XmlPullParserException, IOException;

private String readEntry(XmlPullParser parser, String entryName)
        throws XmlPullParserException, IOException;

private String readText(XmlPullParser parser)
        throws XmlPullParserException, IOException;

private void skip(XmlPullParser parser)
        throws XmlPullParserException, IOException;

public void ExampleParsing()
        throws XmlPullParserException, IOException;
\end{lstlisting}
\subsection{MainActivity.java}
Zawiera definicję klasy \textbf{MainActivity} dziedziczącej po klasie \textbf{AppCompatActivity}.\\\\
Lista pól i metod:
\begin{itemize}
\item\begin{lstlisting}
Questions questions;
\end{lstlisting}
Kontener z załadowanymi pytaniami.
\item\begin{lstlisting}
Questions.Question currentQuestion;
\end{lstlisting}
Obecne pytanie.
\item\begin{lstlisting}
int currentQuestionIndex;
\end{lstlisting}
Index obecnego pytania.
\item\begin{lstlisting}
int score;
\end{lstlisting}
Posiadane punty.
\item\begin{lstlisting}
@BindView(R.id.questionTextView)
TextView questionTextView;

@BindView(R.id.scoreTextView)
TextView scoreTextView;

@BindViews({
R.id.answer1Button,
R.id.answer2Button,
R.id.answer3Button,
R.id.answer4Button
}) List<Button> answerButtons;
\end{lstlisting}
Referencje do elemntów UI.
\item\begin{lstlisting}
@OnClick({
R.id.answer1Button,
R.id.answer2Button,
R.id.answer3Button,
R.id.answer4Button
})
public void answerButton_onClick(Button b);
\end{lstlisting}
Obsługa naciśnięcia guzika.
\item\begin{lstlisting}
protected void onCreate(Bundle savedInstanceState);
\end{lstlisting}
Inicjalizacja aktywności.\\
Sprawdzenie uprawnień do odczytu plików.\\
Ustawienie elementów UI.\\
Załadowanie pytań.
\item\begin{lstlisting}
private void updateQuestion();
\end{lstlisting}
Załadowanie pytanie do \textbf{currentQuestion} o indeksie \textbf{currentQuestionIndex}.
\item\begin{lstlisting}
private void gameOver();
\end{lstlisting}
Wyświetl komunikat o zakończeniu quizu, pokaż ilość punktów i zapytaj czy spróbować jeszcze raz, czy zakończyć działanie aplikacji.
\item\begin{lstlisting}
private void noQuestionFileFound();
\end{lstlisting}
Poinformuj o nie znalezieniu plików z pytaniami i zakończ.
\item\begin{lstlisting}
private final int STORAGE_PERMISSION_CODE = 1;
\end{lstlisting}
\item\begin{lstlisting}
private void requestStoragePermission();
\end{lstlisting}
Wyświetl komunikat o niedostatecznych uprawnieniach do odczytu plików na urządzeniu, poproś o ich ustawienie.\\
Zakończ w przypadku odmowy użytkownika.
\item\begin{lstlisting}
@Override
public void
onRequestPermissionsResult(
int requestCode,
@NonNull String[] permissions,
@NonNull int[] grantResults
);
\end{lstlisting}
Obsłuż wynik zapytania o uprawnienia.
\end{itemize}
\section{Źródła}
\begin{enumerate}
\item Butter Knife\\
\url{https://jakewharton.github.io/butterknife/}
\item Parsing xml\\
\url{https://developer.android.com/training/basics/network-ops/xml.html}
\item Random shuffling of an array\\
\url{https://stackoverflow.com/questions/1519736/random-shuffling-of-an-array#1520212}
\item Formatting strings\\
\url{https://developer.android.com/reference/java/util/Formatter.html}
\item java.util.Random.nextInt() Method\\
\url{https://www.tutorialspoint.com/java/util/random_nextint_inc_exc.htm}
\item Why can't a Java class be declared as static?\\
\url{https://stackoverflow.com/questions/2376938/why-cant-a-java-class-be-declared-as-static}
\item How to make a Java Generic method static?\\
\url{https://stackoverflow.com/questions/4409100/how-to-make-a-java-generic-method-static#4409134}
\item Use of final class in Java\\
\url{https://stackoverflow.com/questions/5181578/use-of-final-class-in-java}
\item Type List vs type ArrayList in Java\\
\url{https://stackoverflow.com/questions/2279030/type-list-vs-type-arraylist-in-java}
\item Rethrowing checked exceptions\\
\url{https://stackoverflow.com/questions/4554230/rethrowing-checked-exceptions#4554253}
\end{enumerate}
Dokument wygenerowano programem pdfTeX 3.14159265-2.6-1.40.16.
\end{document}
