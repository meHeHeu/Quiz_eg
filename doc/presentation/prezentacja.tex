\documentclass[11pt]{beamer}
\usetheme{default}
\usepackage[utf8]{inputenc}
\usepackage[polish]{babel}
\usepackage[T1]{fontenc}
\usepackage{amsmath}
\usepackage{amsfonts}
\usepackage{amssymb}
\author{Alan Andrzejak, Aleksy Dziarmaga, Michał Kloc}
\title{}
%\setbeamercovered{transparent}
\setbeamertemplate{navigation symbols}{}
\date{}
\subject{Domain coloring}
%%%%%%%%%%%%%%%%
\begin{document}
%%%
\begin{frame}
\titlepage
\end{frame}
%%%
\section{Presentation plan}
\begin{frame}
\frametitle{Presentation plan}\pause
\begin{itemize}
\item $\mathbb{C}\mbox{omplex numbers}$
\pause
\item HSV color model
\pause
\item Domain coloring
\end{itemize}
\end{frame}
%%%
\section{Complex numbers - brief description}
\begin{frame}
\frametitle{Complex numbers - brief description}
\begin{block}{definition}\pause
Let $x_1, x_2, y_1, x_2\in\mathbb{R}$. We define the set of complex numbers as cartesian product $\mathbb{R}^2$ with operations:
\pause
\\
Addition:
$$(x_1,y_1)+(x_2,y_2)=(x_1+x_2,y_1+y_2)$$\pause
Multiplication:
$$(x_1,y_1)\cdot(x_2,y_2)=(x_1X_2-y_1y_2,x_1y_2+x_2y_1)$$
\end{block}
\pause
\begin{block}{}
Complex numbers have many interesting applications and properties. They can be used for finding solutions of polynomials, or some differential equations; they are used in electronics to represent alternating current, or to describe rotations. They form field, they have no natural order\dots
\end{block}
\end{frame}
%%%
\section{Algebraic form}
\begin{frame}
\frametitle{Algebraic form}
\begin{block}{}
Each complex number $(x,y)$ can be represented in algebraic form as sum of first coordinate $x$ and the second one $y$ multiplied by $i$ element, which satisfies condition $i^2=-1$:
\pause
$$x+yi$$
\end{block}
\pause
\begin{block}{Graphical representation}
\begin{center}
\includegraphics[width=2cm, keepaspectratio]{graphic_representation_algebraic.png}
\end{center}
\end{block}
\end{frame}
%%%
\section{Polar form}
\begin{frame}
\frametitle{Polar form}
\begin{center}
\includegraphics[width=2cm, keepaspectratio]{graphic_representation_polar.png}
\end{center}
\pause
\begin{block}{Polar form}
Each complex number $z=x+yi$ can be represented in so called polar form, wchich consists of two numbers, positive modulus $|z|=\sqrt{x^2+y^2}$ and angle $\phi\in[0,2\pi)$ (commonly called argument).
\end{block}
\end{frame}
%%%
\section{Color representation}
\frametitle{Color representation models}
\begin{frame}
\begin{itemize}
\item RGB \pause(Red Green Blue)\pause
\item HSV \pause(Hue Saturation Value)
\end{itemize}
\end{frame}
%%%
\section{RGB color model}
\begin{frame}
In RGB color model colors are represented by three numbers - respectivly red, green and blue.
In general integers from range $[0,255]$ are used to represent colors in computer applications, which gives us 16,777,216 different colors.
\begin{center}
\includegraphics[width=3cm, keepaspectratio]{RGB_color_model.png}
\end{center}
\end{frame}
%%%
\section{HSV color model}
\begin{frame}
Like in RGB color model, in HSV colors are represented by three numbers as well. They have different meaning though: hue, saturation and value.\pause
\begin{center}
\includegraphics[width=5cm, keepaspectratio]{HSV_cone.jpg}
\end{center}
\end{frame}
%%%
\section{Domain coloring}
\frametitle{Complex functions}
\begin{frame}
$$f\colon\mathbb{R}\to\mathbb{R}$$\pause
\begin{center}
\includegraphics[width=2cm, keepaspectratio]{f_ex.png}
\end{center}\pause
$$f\colon\mathbb{R}^2\to\mathbb{R}$$\pause
\begin{center}
\includegraphics[width=2cm, keepaspectratio]{mf_ex.png}
\end{center}\pause
$$f\colon\mathbb{C}\to\mathbb{C}$$
$$f\colon\mathbb{R}^2\to\mathbb{R}^2$$\pause
\begin{center}?\end{center}
\end{frame}
%%%
\begin{frame}
\begin{center}
\includegraphics[width=4cm, keepaspectratio]{graphic_representation_polar.png}
\includegraphics[width=4cm, keepaspectratio]{HSV_cone.jpg}
\end{center}
\end{frame}
%%%
\section{END}
\begin{frame}
\Huge
\center
\textbf{END}
\end{frame}
%„HSV cone” autorstwa Created by Wapcaplet - From en wiki. Licencja CC BY-SA 3.0 na podstawie Wikimedia Commons - https://commons.wikimedia.org/wiki/File:HSV_cone.jpg#/media/File:HSV_cone.jpg
%http://macosa.dima.unige.it/mat/calculus/surfacecontours.htm
\end{document}
%%%%%%%%%%%%%%%%
